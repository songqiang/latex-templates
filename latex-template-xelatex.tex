\documentclass[10pt,letterpaper]{article} 

\usepackage{indentfirst}
\usepackage[colorlinks,linkcolor=black]{hyperref}

\usepackage{fontspec}
\setmainfont{FZShuSong-Z01}
% \setmainfont{NSimSun}

\XeTeXlinebreaklocale "zh"
\XeTeXlinebreakskip = 0pt plus 1pt

\linespread{1.2}

\title{论语} \author{} \date{}

\begin{document}
\section{方正书宋}

    子曰:“学而时习之,不亦悦乎?有朋自远方来,不亦乐乎?人不知而不愠,不亦君子乎?”

    有子曰:“其为人也孝悌而好犯上者,鲜矣。不好犯上而好作乱者,未之有也。君子务本,本立而道生。孝悌也者,其为仁之本与?”

\fontspec{NSimSun}
\section{新宋体}
    子曰:“巧言令色,鲜矣仁。”

    曾子曰:吾日三省乎吾身。为人谋而不忠乎?与朋友交而不信乎?传不习乎?

    子曰:道千乘之国,敬事而信,节用而爱人,使民以时。

\fontspec{FZXiHeiI-Z08}
\section{方正细黑}

    子曰:弟子入则孝,出则悌,谨而信,泛爱众而亲仁,行有余力,则以学文。

    子夏曰:贤贤易色,事父母,能竭其力。事君,能致其身。与朋友交,言而有信。虽曰未学,吾必谓之学矣。

    子曰:君子不重则不威,学则不固。主忠信,无友不如己者,过则勿惮改。

\fontspec{FZLiShu-S01}
\section{方正隶书}

    曾子曰:慎终追远,民德归厚矣。

    子禽问于子贡曰:“夫子至于是邦也,必闻其政。求之与?抑与之与?”子贡曰:“夫子温良恭俭让以得之。夫子求之也,其诸异乎人之求之与?”

\fontspec{FZKai-Z03}
\section{方正楷体}

    子曰:父在,观其志。父没,观其行。三年无改于父之道,可谓孝矣。

    有子曰:礼之用,和为贵。先王之道斯为美。小大由之,有所不行。知和而和,不以礼节之,亦不可行也。

\fontspec{FZHei-B01}
\section{微软雅黑}

    有子曰:信近于义,言可复也。恭近于礼,远耻辱也。因不失其亲,亦可宗也。

    子曰:君子食无求饱,居无求安。敏于事而慎于言,就有道而正焉。可谓好学也已。

\fontspec{AR PL UMing CN}
\section{上海宋}

    子贡曰:“贫而无谄,富而无骄。何如?”子曰:“可也。未若贫而乐,富而好礼者也。”子贡曰:“诗云:如切如磋,如琢如磨。其斯之谓与?”子曰:“赐也,始可与言诗已矣。告诸往而知来者。”

    子曰:不患人之不己知,患不知人也。



\end{document}

%%% list avaible font, run 
% fc-list
% fc-list :lang=zh-cn
% fc-list :lang=zh-tw

%%% compile latex file, run  
% xelatex latex-template-xelatex.tex

